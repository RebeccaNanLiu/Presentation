\documentclass[mathserif]{beamer}
\input{macro}
\usepackage{graphicx}
\usepackage{xcolor}
\usepackage[justification=centering]{caption}
\setbeamertemplate{bibliography item}{\insertbiblabel}

\beamertemplatenavigationsymbolsempty

\setbeamerfont{page number in head/foot}{size=\small}
\setbeamertemplate{footline}[frame number]

\usepackage{comment}

\mode<presentation>
{

  	\usetheme{Pittsburgh}

  	%\usetheme[numbers,
	 %pageofpages=of,% String used between the current page and the total page count.
         % bullet=circle,% Use circles instead of squares for bullets.
          %titleline=true,% Show a line below the frame title.
          %]{Singapore}
            
%  \setbeamertemplate{footline}[frame number]
%  \usefonttheme[onlysmall]{structurebold}
%  \setbeamercovered{dynamic}
% \setbeamercovered{transparent}

\usebackgroundtemplate{\includegraphics[height=\paperheight]{template}}

\setbeamertemplate{frametitle}{
	\vskip30pt
 	\usebeamercolor[blue!60!green]{frametitle}
  	\centering
  	\insertframetitle
	\vskip5pt
	\color{blue!60!green}{\hrule}
	}
}

\setbeamercolor{math text}{fg=orange!80!black}
\newcommand{\hl}[1]{\textcolor{red!80!black}{\underline{#1}}}


\title[short title]{Random GP Forest}
\author[short presentator]{Raphael, Andreas, Nan, Dragomir}
\institute[CAMP]{Computer Vision Group \\
Technische Universit\"at M\"unchen}
\date[]{27.07.2015}


\begin{document}

%---------------------------------------------------------------------------------------

\begin{frame}
\titlepage
\index{•}\end{frame}

%---------------------------------------------------------------------------------------
\begin{frame}
\frametitle{Outline}

\begin{itemize}
\item Problem 
\item Objective
\item Milestone
\item Implementation
\item Experiment and Evaluation
\item Conclusion and Discussion
\end{itemize}

\end{frame}
%---------------------------------------------------------------------------------------
\begin{frame}
\frametitle{Problem}

\begin{columns}[t]
\begin{column}{0.5\linewidth}
\begin{itemize}
\item Gaussian Process (GP)
\begin{itemize}
\item high training time complexity	
\end{itemize}
\vspace{0.5cm}
\item Random Forest (RDF)
\begin{itemize}
\item 	moderate accuracy rate
\end{itemize}
%\vspace{0.5cm}
\end{itemize}
\end{column}

\begin{column}{0.5\linewidth}
\begin{figure}[t]
\centering
\scriptsize{Table 1: Computational complexity of all three methods considered: n denotes the number of training examples, L refers to the maximum number of examples a GP classifier is learned within a leaf, T is the number of decision trees in the forest$_{[1]}$.}
\includegraphics[width=4cm,height=3cm]{time}
\end{figure}
\end{column}
\end{columns}

\end{frame}
%---------------------------------------------------------------------------------------
\section{Introduction}
\begin{frame}
\frametitle{Objective}

\begin{columns}[t]
\begin{column}{0.5\linewidth}
\begin{itemize}
\item Combining RDF and GP (RDF-GP)
\begin{itemize}
\item enable accurate classification in large-scale settings
\item GP: state-of-the-art recognition performance
\item RDF:  applied to large-scale dataset
\end{itemize}
\end{itemize}
\end{column}

\begin{column}{0.5\linewidth}
\begin{figure}[t]
\centering
\includegraphics[width=1.1\linewidth]{fig1}

\scriptsize{Figure 1: RDF is used to cluster the data in a supervised manner and a GP classifier is used to separate classes in each leaf$_{[1]}$.}
\label{fig1}
\end{figure}
\end{column}
\end{columns}

\end{frame}
%---------------------------------------------------------------------------------------
\begin{frame}
\frametitle{Milestones}

\begin{itemize}
\item Get GP classifier working \{3 weeks\}
\vspace{0.5cm}
\item Get online RDF working \{2 weeks\}
\vspace{0.5cm}
\item Combine GP and online RDF \{5 weeks\}
\end{itemize}

\end{frame}
%---------------------------------------------------------------------------------------
\begin{frame}
\frametitle{Implementation}

\begin{itemize}
\item language: C++
\item IDE : Ubantu ......
\end{itemize}

\end{frame}
%---------------------------------------------------------------------------------------


\begin{frame}
\frametitle{Implementation}

\begin{itemize}
\item GP [Raphael Dragomir]
\begin{itemize}
\item library:
\item challenge:
\item solutions:
\end{itemize}
\end{itemize}

\end{frame}
%---------------------------------------------------------------------------------------
\begin{frame}
\frametitle{Implementation}

\begin{columns}[t]
\begin{column}{0.5\linewidth}
\begin{itemize}
\item  RDF [Andreas Nan]
\begin{itemize}
\item OpenCV library, faild! 
\item Challenge: online version
\item Solution: Open source code from Amir Saffari et.al.[3]
\end{itemize}
\end{itemize}
\end{column}

\begin{column}{0.5\linewidth}
\begin{figure}[t]
\centering
\includegraphics[width=4.5cm,height=4.5cm]{rdf_perf_onoff}

\scriptsize{Figure 2: Classification error with respect to the ratio of labeled samples for off-line and on-line training with increasing number of training samples$_{[3]}$.}
\label{fig2}
\end{figure}
\end{column}
\end{columns}

\end{frame}
%---------------------------------------------------------------------------------------
\begin{frame}
\frametitle{Implementation}

\begin{itemize}
\item Combining GP with online RDF
\begin{itemize}
\item Challenge 1:  when to train GP?
\item Solution:  buffer
\item Challenge 2:  multiclassification?
\item Solution:  train n GPCs for each leaf node
\end{itemize}
\end{itemize}

\end{frame}
%---------------------------------------------------------------------------------------
\begin{frame}
\frametitle{Implementation}

\begin{itemize}
\item What the system is capable of?
\begin{itemize}
\item GP classification
\item RDF classification
\item RDF-GP classification
\end{itemize}
\end{itemize}

\end{frame}
%---------------------------------------------------------------------------------------


\begin{frame}
\frametitle{Experiments and Evaluation}
\begin{itemize}
\item Data: RGB-D Dataset$_{[2]}$ 
\end{itemize}

\end{frame}
%---------------------------------------------------------------------------------------
\begin{frame}
\frametitle{Conclusion and Discussion}
\begin{itemize}
\item Our Contributions:
\begin{itemize}
\item combined GP and RDF
\item implemented online version
\item obtained multi classifier
\end{itemize}
\end{itemize}

\begin{itemize}
\item Advantages:
\begin{itemize}
\item good accuracy maintained
\item substantially faster than the full GP classifier
\end{itemize}
\end{itemize}


\begin{itemize}
\item Future Work:
\begin{itemize}
\item parameter optimization 
\item using GPU for speed optimization
\end{itemize}
\end{itemize}


\end{frame}
%---------------------------------------------------------------------------------------
\begin{frame}%[allowframebreaks]
\frametitle{References}
\fontsize{6}{9}
\begin{thebibliography}{99}
\bibitem{1}
B. Fröhlich, E. Rodner, M. Kemmler, and J. DenzlerI,	 “Large-Scale Gaussian Process Classification Using Random Decision Forests, ” ISSN 1054-6618, Pattern Recognition and Image Analysis, 2012, Vol. 22, No. 1, pp. 113–120, 2012. 

\bibitem{2}
Kevin Lai, Liefeng Bo, Xiaofeng Ren, and Dieter Fox.  A Large-Scale Hierarchical Multi-View RGB-D Object Dataset. http://rgbd-dataset.cs.washington.edu

\bibitem{3}
Amir Saffari, Christian Leistner, Jakob Santner, Martin Godec, and Horst Bischof, "On-line Random Forests," 3rd IEEE ICCV Workshop on On-line Computer Vision, 2009.
\end{thebibliography}
\end{frame}
%---------------------------------------------------------------------------------------
\begin{frame}
\centering
\begin{figure}[b]
\includegraphics[width=0.9\linewidth]{thankyou}
\end{figure}
\end{frame}
\end{document}



