% Template created by Robert Maier, 2013
\documentclass[t,plaincaption]{beamer}

\mode<presentation>
{
	\usepackage{theme_cvpr/beamerthemeCVPR}
	\setbeamercovered{transparent}
}

\usepackage{verbatim}

% Use xelatex to use TTF fonts 
\usepackage{fontspec}
\setsansfont{Arial}

% set the bibliography style
%\bibliographystyle{abbrv}
\bibliographystyle{apalike}

% set document information
\def\titleEn{Random GP Forest}
\def\authorName{Raphael Dümig, Andreas Wiedemann, Nan Liu, Dragomir Nikolic\\
\vspace{0.5cm}
Supervisor: Dr. habil. Rudolph Triebel}
\title[\titleEn]{\titleEn}
\author[Raphael Dümig, Andreas Wiedemann, Nan Liu, Dragomir Nikolic: \titleEn]{\authorName}
\date{July 27, 2015}


\begin{document}

\frame{
\titlepage 
}

\frame{
\frametitle{Outline}

\tableofcontents
}


\section{Problem}
\frame{
\frametitle{Problem}

\begin{columns}[t]
\begin{column}{0.45\linewidth}
\begin{itemize}
\item Gaussian Process (GP)
\begin{itemize}
\item high training time complexity	
\end{itemize}
\vspace{0.5cm}
\item Random Forest (RDF)
\begin{itemize}
\item 	moderate accuracy rate
\end{itemize}
%\vspace{0.5cm}
\end{itemize}
\end{column}

\begin{column}{0.65\linewidth}
\begin{figure}[t]
\centering

\begin{center}
\hspace{-3cm}\includegraphics[width=3cm,height=1cm]{time}
\end{center}
\vspace{0.3cm}
\scriptsize{Table 1: Computational complexity: n denotes the number of training examples, L refers to the maximum number of examples a GP classifier is learned within a leaf, T is the number of decision trees in the forest$_{[1]}$.}

\end{figure}
\end{column}
\end{columns}
}


\section{Objective}
\frame{
\frametitle{Objective}

\begin{columns}[t]
\begin{column}{0.5\linewidth}
\begin{itemize}
\item Combining RDF and GP (RDF-GP)
\begin{itemize}
\item enable accurate classification in large-scale settings
\item GP: state-of-the-art recognition performance
\item RDF:  applied to large-scale dataset
\end{itemize}
\end{itemize}
\end{column}

\begin{column}{0.5\linewidth}
\begin{figure}[t]
\centering
\hspace{-2.5cm}\includegraphics[width=0.5\linewidth]{fig1}

%\vspace{0.5cm}
\scriptsize{Figure 1: RDF is used to cluster the data in a supervised manner and a GP classifier is used to separate classes in each leaf$_{[1]}$.}
\label{fig1}
\end{figure}
\end{column}
\end{columns}
}


\section{Milestones}
\frame{
\frametitle{Milestones}

\begin{itemize}
\item Get GP classifier working \{3 weeks\}
\vspace{0.5cm}
\item Get online RDF working \{2 weeks\}
\vspace{0.5cm}
\item Combine GP and online RDF \{5 weeks\}
\end{itemize}
}

\section{Implementation}
\frame{
\frametitle{Implementation}

\begin{itemize}
\item language: C++
\item IDE : Ubantu ......
\end{itemize}

}

%\section{Implementation}
\frame{
\frametitle{Implementation}

\begin{itemize}
\item GP [Raphael Dümig, Dragomir Nikolic]
\begin{itemize}
\item library:
\item challenge:
\item solutions:
\end{itemize}
\end{itemize}

}

%\section{Implementation}
\frame{
\frametitle{Implementation}

\begin{itemize}
\item  RDF [Andreas Wiedemann, Nan Liu]
\begin{itemize}
\item OpenCV library\ \ \ \ off-line version
\item Amir Saffari [3]\ \ \ \ \ on-line version
\end{itemize}
\end{itemize}
}
\frame{
\frametitle{Implementation}
\begin{columns}[t]
\begin{column}{0.5\linewidth}
\begin{figure}[t]
\centering
\vspace{2cm}
\hspace{-2cm}\includegraphics[width=2cm,height=3cm]{rdfalg}

\scriptsize{Figure 2: Online RDF algorithm$_{[3]}$.}
\label{fig2}
\end{figure}
\end{column}

\begin{column}{0.5\linewidth}
\begin{figure}[t]
\centering
\vspace{2cm}
\hspace{-2.8cm}\includegraphics[width=3cm,height=1.5cm]{rdf_perf_onoff}

\scriptsize{Figure 3: Classification error with respect to the ratio of labeled samples for off-line and on-line training with increasing number of training samples$_{[3]}$.}
\label{fig3}
\end{figure}
\end{column}
\end{columns}

}

%\section{Implementation}
\frame{
\frametitle{Implementation}

\begin{itemize}
\item Combining GP with online RDF
\begin{itemize}
\item Challenge 1:  when to train GP?
\item Solution:  buffer
\item Challenge 2:  multiclassification?
\item Solution:  train n GPCs for each leaf node
\end{itemize}
\end{itemize}
}


%\section{Implementation}
\frame{
\frametitle{Implementation}

\begin{itemize}
\item What the system is capable of?
\begin{itemize}
\item GP classification
\item RDF classification
\item RDF-GP classification
\item binary and multi classification
\item online version
\end{itemize}
\end{itemize}
}

\section{Experiments and Evaluation}
\frame{
\frametitle{Experiments and Evaluation}
\begin{itemize}
\item Data: RGB-D Dataset$_{[2]}$ 
\end{itemize}
% GNUPLOT: LaTeX picture
\setlength{\unitlength}{0.240900pt}
\ifx\plotpoint\undefined\newsavebox{\plotpoint}\fi
\sbox{\plotpoint}{\rule[-0.200pt]{0.400pt}{0.400pt}}%
\begin{picture}(1181,590)(0,0)
\sbox{\plotpoint}{\rule[-0.200pt]{0.400pt}{0.400pt}}%
\put(110.0,82.0){\rule[-0.200pt]{4.818pt}{0.400pt}}
\put(90,82){\makebox(0,0)[r]{$0$}}
\put(1100.0,82.0){\rule[-0.200pt]{4.818pt}{0.400pt}}
\put(110.0,159.0){\rule[-0.200pt]{4.818pt}{0.400pt}}
\put(90,159){\makebox(0,0)[r]{$0.2$}}
\put(1100.0,159.0){\rule[-0.200pt]{4.818pt}{0.400pt}}
\put(110.0,236.0){\rule[-0.200pt]{4.818pt}{0.400pt}}
\put(90,236){\makebox(0,0)[r]{$0.4$}}
\put(1100.0,236.0){\rule[-0.200pt]{4.818pt}{0.400pt}}
\put(110.0,312.0){\rule[-0.200pt]{4.818pt}{0.400pt}}
\put(90,312){\makebox(0,0)[r]{$0.6$}}
\put(1100.0,312.0){\rule[-0.200pt]{4.818pt}{0.400pt}}
\put(110.0,389.0){\rule[-0.200pt]{4.818pt}{0.400pt}}
\put(90,389){\makebox(0,0)[r]{$0.8$}}
\put(1100.0,389.0){\rule[-0.200pt]{4.818pt}{0.400pt}}
\put(110.0,466.0){\rule[-0.200pt]{4.818pt}{0.400pt}}
\put(90,466){\makebox(0,0)[r]{$1$}}
\put(1100.0,466.0){\rule[-0.200pt]{4.818pt}{0.400pt}}
\put(278.0,82.0){\rule[-0.200pt]{0.400pt}{4.818pt}}
\put(278,41){\makebox(0,0){RDF}}
\put(278.0,446.0){\rule[-0.200pt]{0.400pt}{4.818pt}}
\put(615.0,82.0){\rule[-0.200pt]{0.400pt}{4.818pt}}
\put(615,41){\makebox(0,0){RDF-GP}}
\put(615.0,446.0){\rule[-0.200pt]{0.400pt}{4.818pt}}
\put(952.0,82.0){\rule[-0.200pt]{0.400pt}{4.818pt}}
\put(952,41){\makebox(0,0){GP}}
\put(952.0,446.0){\rule[-0.200pt]{0.400pt}{4.818pt}}
\put(110.0,82.0){\rule[-0.200pt]{0.400pt}{92.506pt}}
\put(110.0,82.0){\rule[-0.200pt]{243.309pt}{0.400pt}}
\put(1120.0,82.0){\rule[-0.200pt]{0.400pt}{92.506pt}}
\put(110.0,466.0){\rule[-0.200pt]{243.309pt}{0.400pt}}
\put(615,528){\makebox(0,0){Accuracies of classifiers}}
\put(194,82){\rule{40.953pt}{30.8352pt}}
\put(194.0,82.0){\rule[-0.200pt]{0.400pt}{30.594pt}}
\put(194.0,209.0){\rule[-0.200pt]{40.712pt}{0.400pt}}
\put(363.0,82.0){\rule[-0.200pt]{0.400pt}{30.594pt}}
\put(531,82){\rule{40.7121pt}{47.4573pt}}
\put(194.0,82.0){\rule[-0.200pt]{40.712pt}{0.400pt}}
\put(531.0,82.0){\rule[-0.200pt]{0.400pt}{47.216pt}}
\put(531.0,278.0){\rule[-0.200pt]{40.471pt}{0.400pt}}
\put(699.0,82.0){\rule[-0.200pt]{0.400pt}{47.216pt}}
\put(868,82){\rule{40.7121pt}{65.043pt}}
\put(531.0,82.0){\rule[-0.200pt]{40.471pt}{0.400pt}}
\put(868.0,82.0){\rule[-0.200pt]{0.400pt}{64.802pt}}
\put(868.0,351.0){\rule[-0.200pt]{40.471pt}{0.400pt}}
\put(1036.0,82.0){\rule[-0.200pt]{0.400pt}{64.802pt}}
\put(868.0,82.0){\rule[-0.200pt]{40.471pt}{0.400pt}}
\put(110.0,82.0){\rule[-0.200pt]{0.400pt}{92.506pt}}
\put(110.0,82.0){\rule[-0.200pt]{243.309pt}{0.400pt}}
\put(1120.0,82.0){\rule[-0.200pt]{0.400pt}{92.506pt}}
\put(110.0,466.0){\rule[-0.200pt]{243.309pt}{0.400pt}}
\end{picture}


}

 
\section{Conclusion and Future Work}
\frame{
\frametitle{Conclusion and Future Work}

\begin{itemize}
\item Our Contributions:
\begin{itemize}
\item combined GP and RDF
\item implemented online version
\item obtained multi classifier
\end{itemize}
\end{itemize}

\begin{itemize}
\item Advantages:
\begin{itemize}
\item good accuracy maintained
\item substantially faster than the full GP classifier
\end{itemize}
\end{itemize}


\begin{itemize}
\item Future Work:
\begin{itemize}
\item parameter optimization 
\item using GPU for speed optimization
\end{itemize}
\end{itemize}

}

\frame[allowframebreaks]{
\frametitle{Bibliography}
	\tiny
	\bibliography{bibliography} 
\begin{thebibliography}{99}
\bibitem{1}
B. Fröhlich, E. Rodner, M. Kemmler, and J. DenzlerI,	 “Large-Scale Gaussian Process Classification Using Random Decision Forests, ” ISSN 1054-6618, Pattern Recognition and Image Analysis, 2012, Vol. 22, No. 1, pp. 113–120, 2012. 

\bibitem{2}
Kevin Lai, Liefeng Bo, Xiaofeng Ren, and Dieter Fox.  A Large-Scale Hierarchical Multi-View RGB-D Object Dataset. http://rgbd-dataset.cs.washington.edu

\bibitem{3}
Amir Saffari, Christian Leistner, Jakob Santner, Martin Godec, and Horst Bischof, "On-line Random Forests," 3rd IEEE ICCV Workshop on On-line Computer Vision, 2009.
\end{thebibliography}
}

\end{document}
